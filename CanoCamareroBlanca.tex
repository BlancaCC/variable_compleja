\documentclass[12pt]{article}
\usepackage[utf8]{inputenc}
\usepackage[spanish]{babel}
\usepackage{amsmath} 
\usepackage{amssymb} % utlizar mathbb


\newenvironment{micaja}
{
    \begin{center}
    \begin{tabular}{|p{0.9\textwidth}|}
    \hline\\
    }   
    {   
    \\\\\hline
    \end{tabular} 
    \end{center}
    }



\title{Prueba de variable compleja}
\author{Blanca Cano Camarero}

\begin{document}
\begin{titlepage}
\maketitle
\tableofcontents
\end{titlepage}

\section{Ejercicio 1}

\begin{micaja}
    Estudiar la convergencia puntual, absoluta y uniforme de la serie $\sum_{n \geq 0} f_n$ donde
    $$f_n(z) = \left( \frac{z^2 - i}{z^2+i} \right)^n \forall z \in \mathbb C \setminus  \left\{ \pm  \frac{-1-i}{\sqrt{2}} \right\}$$
\end{micaja}


Llamando $\phi(z) = \frac{z^2 - i}{z^2+i}$ tenemos que nuestra serie a estudiar 
 es $\sum_{n \geq 0} \phi(z)^n$, una serie geométrica. \par
 Sabemos que toda serie geométrica $\sum_{n \geq 0} w^n$: 

 \begin{itemize}
     \item Será absolutamente convergente para $w \in D(0,1).$
     \item Uniformemente convergente para $w \in k$ con k un compacto de $D(0,1).$
 \end{itemize}
Por lo tanto nos interesa estudiar el conjunto de puntos $\Omega$ para el que está definida la
serie y que cumpla que  $\forall z \in \Omega \subseteq \mathbb C \setminus  \left\{ \pm  \frac{-1-i}{\sqrt{2}} \right\}$ 
se tenga que  $|\phi(z)|<1$. \paragraph{}
 Deducción  analítica de $\Omega$.

Denotaremos todo número complejo z como $z = x+yi$ con $x,y \in \mathbb R.$

$$z^2-i = (x^2 - y^2) + (2xy -1)i$$
$$z^2+i = (x^2 - y^2)+(2xy + 1)i$$

Y sus respectivos módulos son: 

$$|z^2-i| = \sqrt{ (x^2 - y^2)^2 + (2xy -1)^2}$$
$$|z^2+i| = \sqrt{ (x^2 - y^2)^2 + (2xy +1)^2}$$

Por tanto 

$$\left| \frac{z^2-i}{z^2+i} \right| <1 \Longleftrightarrow \frac{|z^2-i|}{|z^2+i|}<1 \Longleftrightarrow \sqrt{ (x^2 - y^2)^2 + (2xy -1)^2} < \sqrt{ (x^2 - y^2)^2 + (2xy +1)^2}$$
$$\Longleftrightarrow(2xy-1)^2 <(2xy+1)^2 \Longleftrightarrow 0 < xy$$
Por tanto concluimos que esto válido para puntos del primer y tercer cuadrante en que esté definida la función, 
es decir 
$$\Omega = \{ x+yi  :   x,y \in \mathbb R^+ \vee x,y \in \mathbb R^-\} \setminus  \left\{ \pm  \frac{-1-i}{\sqrt{2}} \right\}.$$

\begin{itemize}
\item \textbf{Convergencia absoluta}:  
La serie converge absolutamente en $\Omega.$ 
\item \textbf{Convergencia uniforme}:
 La serie será uniformemente convergente en cualquier compacto de $\Omega.$
\item \textbf{Convergencia puntual}: Convergencia absoluta implica puntual, por tanto converge puntualmente en $\Omega.$ 
\item Para puntos fuera de $\Omega$ la serie diverge. 
\end{itemize}

\newpage
\section{Ejercicio 2}  
\begin{micaja}

    Estudiar la derivabilidad  de las funciones $f,g: \mathbb C \rightarrow \mathbb C$
\end{micaja}

\subsection{$f(z)=z^2 e^{\bar z}$}

$z^2$ es un polinomio que es derivable en todo $\mathbb C$ y solo se anula en $0.$ 

Si despejamos la exponencial llegamos a la siguiente ecuación:
$$\frac{f(z)}{z^2} = e^{\bar z}.$$
La cual nos va ayudar a conocer los puntos en los que $f$ NO es derivable,
pues por la regla de derivación de la división tenemos que si $f \in \mathcal H(\Omega)$ con $\Omega \subseteq \mathbb C$ entonces 
$$\frac{f(z)}{z^2} =  e^{\bar z} \in \mathcal H(\Omega \cap \mathbb C^*)$$ 
(Nótese que se ha excluido z=0).   \paragraph{}

Estudiemos la derivabilidad de $e^{\bar z}$. \paragraph{}

Para simplificar la notación denotaré a $x = Re z$, $y = Im z$ Con $z \in \Omega \cap \mathbb C^*$
$$e^{\bar z} = e^x (cos(-y) + i sin(-y)) = e^x cos(y) - i e^x sin(y)$$
$$\frac{\partial}{\partial x} Re e^{\bar z} = e^x cos(y)$$
$$\frac{\partial}{\partial y} Re e^{\bar z} = - e^x sin(y)$$
$$\frac{\partial}{\partial x} Im e^{\bar z} = - e^x sin(y)$$
$$\frac{\partial}{\partial y} Im e^{\bar z} = - e^x cos(y)$$

Lo puntos derivables tienen que cumplir las ecuaciones de Cauchy-Rieman:
$$\frac{\partial}{\partial x} Re e^{\bar z} = \frac{\partial}{\partial y} Im e^{\bar z} 
\wedge
\frac{\partial}{\partial y} Re e^{\bar z} = - \frac{\partial}{\partial x} Im e^{\bar z}.$$

Esto es un sistema incompatible, por tanto $e^{\bar z} \notin \mathcal H( \mathbb C^* \cap \Omega)$ 
lo que implica que $f(z) \notin \mathcal H(\Omega)$ con $\Omega$ cualquier conjunto contenido en $\mathbb C^*.$

Estudiamos el caso que nos faltaba por estudiar su derivabilidad $z=0$

$$f'(0) = lim_{z \rightarrow 0} \frac{f(z)-f(0)}{z-0} = lim_{z \rightarrow 0} \frac{z^2 e^{\bar z}}{z} = 0$$

Con lo que concluimos que $f$ es derivable en el 0.


\subsection{$g(z) = sin(z) f(z)$}
Razonaremos de la misma manera: \\
Despejamos $ \frac{g(z)}{sin(z)} = f(z)$
Los puntos que tendremos que estudiar de manera particular 
serán: 
\begin{itemize}
\item Para los que $sin(z) = 0$, estos son los de la forma $z = \pi k$ con $k$ entero.
\item El $z=0$ ya que aquí $f(z)$ es derivable. \paragraph{}
\end{itemize}
Llamaré $\Lambda = \pi\mathbb Z \cup \{0\}$ al conjunto de estos puntos. \paragraph{}

Puesto que $f(z)$ no es derivable en $\mathbb C \setminus \Lambda$
tampoco $g(z)$ lo será ahí. 

Analizaremos ahora la derivabilidad en $\Lambda$ por la definición de derivada. 

$\forall a \in \pi\mathbb Z$ se tiene 
$$lim_{z \longrightarrow a} \frac{g(z)-g(a)}{z-a} = f(a) lim_{z \longrightarrow a} \frac{sin(z)}{z-a}$$
Si escribios $sin(z)$ como el polinomio de taylor centrado en a tenemos: 

$$f(a) lim_{z \longrightarrow a} \frac{sin(z)}{z-a} = f(a) lim_{z \longrightarrow a}
\frac{1}{z-a} (sin(a) + cos(a)(z-a) + \frac{- sin(a)}{2!}(z-a)^2 +\dots )$$
 $$=f(a) lim_{z \longrightarrow a}
  (\frac{sin(a)}{z-a} + cos(a)(z-a)^0 + \frac{- sin(a)}{2!}(z-a)^{2-1} + \dots )$$
  $$=f(a) lim_{z \longrightarrow a}
  (0+ cos(a) + \frac{- sin(a)}{2!}(z-a)\dot )= f(a)cos(a)$$

Que tine límite y por tanto existe la derivada en esos puntos. 

Veamos ahora el caso $z=0$. 

$$lim_{z \longrightarrow a} \frac{g(z)-g(0)}{z-0} = lim_{z \longrightarrow a} \frac{sin(z)}{z}$$
Que si volvemos a escribir $sin(z)$ como el polinomio de taylor centradon en 0 tenemos:

$$lim_{z \longrightarrow 0} \frac{g(z)-g(0)}{z-0} = cos(0).$$ 

Concluimos por tanto que $g \in \mathcal H(\Lambda).$





\newpage
\section{Ejercicio 3}  
\begin{micaja}
    Calcular $$\int_{C(0,1)} \frac{cos(z)}{z(z-2)^2}dz$$
\end{micaja}

Sea $\Omega = D(0,r)$ con $1<r<2$ un abierto. Se tiene que $cos(z) \in \mathcal H(\mathbb C)$ 
y que $(z-2)^2 \in \mathcal H(\mathbb C)$  y además no se anula en $\Omega.$ 
Por tanto definimos $f(z) = \frac{cos(z)}{(z-2)^2}$ y esta función es holomorfa en
$\Omega.$ \paragraph{}

Tenemos además que $\bar{C}(0,1) \subseteq \Omega$ y con esto se cumplen todas las hipótesis
necesarias para poder aplicar la fórmula 
de Cauchy para la circunferencia:  

$$\int_{C(0,1)} \frac{cos(z)}{z(z-2)^2}dz = 
\int_{C(0,1)} \frac{f(z)}{z-0}dz = 2\pi i f(0) = \frac{\pi i}{2}$$

\newpage

\section{Ejercicio 4}

\begin{micaja}
    Sean $a,b  \in \mathbb C$ y sea $R >0$ de modo que $R>max\{|a|,|b|\}.$
    Probar que, si $f$ es una función entera, se tiene que:

    $$\int_{C(0,R)} \frac{f(z)}{(z-a)(z-b)}dz = 2\pi i \frac{f(b)-f(a)}{b-a}.$$
    
    Deducir que toda función entera y acotada es constante (Teorema de Liouville).
\end{micaja}

\subsection{Igualdad}
Sean $a,b  \in \mathbb C$ dos elementos distintos cualesquiera y $R>max\{|a|,|b|\}.$ 
$$ 2\pi i \frac{f(b)-f(a)}{b-a} = \frac{1}{b-a}(2\pi i f(b) - 2\pi i f(a))$$

Y como $f$es una función entera se cumplen las hipótesis de la fórmula de Cauchy para la circunferencia, 
lo que nos permite escribir: 

$$\frac{1}{b-a}(2\pi i f(b) - 2\pi i f(a)) = \frac{1}{b-a}\left( \int_{C(0,R)} \frac{f(z)}{z-b}dz - \int_{C(0,R)} \frac{f(z)}{z-a}dz\right)$$
Aplicamos la aditividad de la integral, suma fraciones y sacar factor común $f(z)$: 
$$=\frac{1}{b-a} \int_{C(0,R)} \frac{f(z)(z-a)-f(z)(z-b)}{(z-b)(z-a)}dz =
\frac{1}{b-a} \int_{C(0,R)} \frac{f(z)(b-a)}{(z-b)(z-a)}dz$$
y finalmente por linealidad de la integral nos queda lo que buscábamos: 
$$=\int_{C(0,R)} \frac{f(z)}{(z-a)(z-b)}dz.$$

\subsection{Teorema de Liouville}
Sean $a,b  \in \mathbb C$ dos elementos cualesquiera distintos y $R >0$ de modo que $R>max\{|a|,|b|\}.$
Como $f$ es entera el ejercicio anterior nos da la la siguiente igualdad: 
$$\int_{C(0,R)} \frac{f(z)}{(z-a)(z-b)}dz = 2\pi i \frac{f(b)-f(a)}{b-a}.$$
Lo que implica que 
\begin{equation}
\left|\int_{C(0,R)} \frac{f(z)}{(z-a)(z-b)}dz \right|= \left| 2\pi i \frac{f(b)-f(a)}{b-a}\right|.
\end{equation}
Por otro lado tenemos
$$\left|\int_{C(0,R)} \frac{f(z)}{(z-a)(z-b)}dz \right| \leq 
 \int_{C(0,R)} \frac{|f(z)|}{|(z-a)(z-b)|}dz$$

Por estar acotada la función entera $\exists M >0$ tal que $|f(z)|\leq M$. 
Y además aplicando desigualdades triangulares: 
$|z-a| \geq ||z|-|a|| = |R-|a|| \in \mathbb R$

De donde deducimos sacando escalar

$$\leq\int_{C(0,R)} \frac{M}{|R-|a|| |R-|b||}dz = \frac{M}{(R-|a|)(R-|b|)}.$$
Y como R puede ser cualquier real siempre que $R>max\{|a|,|b|\},$ 
tenemos que para cualquier $\varepsilon >0$ siempre se va a poder conseguir un radio 
para el cual (volviendo a la ecuación (1))

$$\left|\int_{C(0,R)} \frac{f(z)}{(z-a)(z-b)}dz  \right| = \left|2\pi i \frac{f(b)-f(a)}{b-a}\right| < \varepsilon.$$
y puesto que $a,b$ son dos número complejos distintos cualesquiera; concluimos que
sea cuales sean $a,b \in \mathbb C$ distintos, sus imágenes se pueden 
 acercar todo lo que queramos  $|f(a)-f(b)| < \varepsilon_2$ o lo que es lo mismo, 
la función $f$ es constante. 

\newpage


\section{Ejercicio 5}

\begin{micaja}
    
    Sea $\emptyset \neq \Omega = \mathring{\Omega} \subset \mathbb C$ y sean $g,g_n:\Omega \rightarrow \mathbb C$ para cada $n\in \mathbb N.$
    Probar que $\{g_n\}$ converge uniformemente a g en cada compacto de $\Omega$ si, y solo si, para cada $a \in \Omega$ existe
    un entorno de $a$ en el que $\{g_n\}$ converge uniformemente a g.
\end{micaja}
\subsection{Condición suficiente}
Por ser $\Omega$ un abierto para cualquiera de sus puntos $a$ podemos encontrar un disco $D(a,r)$ que quede contenido en $\Omega.$
Consideremos ahora $\bar D (a,\frac{r}{2}) \subset D(a,r)$  que es un compacto de $\Omega$ y por hipótesis ${g_n}$ convergirá
uniformemente.

Como $a$ era un punto cualquiera hemos probado que para todo punto existe un entorno en que 
${g_n}$ converge uniformemente.

\subsection{Condición necesaria}
Sea $K$ un compacto cualquiera de $\Omega.$
Como para cada $a \in K$ existe un entorno de $a$ en el que $\{g_n\}$ converge uniformemente a g.
Podemos escribir $K$ como unión de abiertos en los que converge uniformemente.
$$K= \cup_{ a \in K} D(a,r_a)$$ 
Donde $D(a,r_a)$ es un entorno abierto en el que hay convergencia uniformemente.

Como además estamos en un compacto existirá un subrecubrimiento finito, 
$\Lambda \subset \Omega$ finito. 

$$K \subset \cup_{ a \in \Lambda} D(a,r_{a})$$ 

Como para cada entorno del subrecubrimiento converge uniformemente tenemos que sea cual sea el 
$a\in \Lambda$
existirá $n_0^{a}  \in \mathbb N$ tal que para todo $n^a>n_0^a$  se cumple que
$$|g_{n^a}(z) - g(z)|< \varepsilon \quad \forall z \in D(a,r_a)$$

Como $\Lambda$ es finito existe $n_0$, el  máximo de los $n_0^a$ anteriores. Entontes concluiremos con que
para todo $\varepsilon >0$ existe $n_0$ para que sea cual seal el $n>n_0$ natura se cumpla que
$$|g_{n}(z) - g(z)|< \varepsilon \quad \forall z \in \cup_{ a \in \Lambda} D(a,r_a) \supseteq K$$

Es decir, que hay convergencia uniforme en ese compacto $k$.
Con lo que hemos probado lo que buscábamos. 

\end{document}
